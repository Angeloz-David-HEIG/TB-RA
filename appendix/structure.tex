\chapter{Structure}

La structure du rapport proposé dans ce modèle n'est qu'une proposition. Vous devez savoir l'adapter à votre projet et à vos besoins. Pour vous aider, voici quelques structures classiques que vous pouvez suivre pour rédiger votre rapport de bachelor.

\section{Structure Classique}

\begin{enumerate}
    \item Introduction
          \begin{itemize}
              \item Contexte général
              \item Problématique
              \item Objectifs de l'étude
              \item Hypothèses de travail
              \item Structure du rapport
          \end{itemize}

    \item Revue de littérature / État de l'art
          \begin{itemize}
              \item Concepts clés
              \item Travaux antérieurs
              \item Limites des recherches existantes
          \end{itemize}

    \item Méthodologie
          \begin{itemize}
              \item Description du cadre d'étude
              \item Matériaux et outils
              \item Méthodes expérimentales / Modélisation
              \item Procédures de collecte de données
          \end{itemize}

    \item Résultats
          \begin{itemize}
              \item Analyse des données
              \item Présentation des résultats
              \item Validation des résultats
          \end{itemize}

    \item Discussion
          \begin{itemize}
              \item Interprétation des résultats
              \item Comparaison avec les études précédentes
              \item Limites de l'étude
          \end{itemize}

    \item Conclusion et perspectives
          \begin{itemize}
              \item Synthèse des résultats
              \item Recommandations
              \item Perspectives de recherche
          \end{itemize}

\end{enumerate}

\section{Ingénierie appliquée}

\item Introduction générale
\begin{itemize}
    \item Contexte industriel
    \item Justification du projet
    \item Objectifs spécifiques
    \item Plan du rapport
\end{itemize}
\item Analyse des besoins
\begin{itemize}
    \item Cahier des charges
    \item Spécifications techniques
    \item Analyse des contraintes
\end{itemize}
\item Conception et développement
\begin{itemize}
    \item Choix des matériaux et des technologies
    \item Design et architecture système
    \item Implémentation technique
\end{itemize}
\item Tests et validation
\begin{itemize}
    \item Méthodologie de test
    \item Résultats expérimentaux
    \item Évaluation des performances
\end{itemize}
\item Discussion et recommandations
\begin{itemize}
    \item Analyse critique des résultats
    \item Propositions d'amélioration
\end{itemize}
\item Conclusion
\begin{itemize}
    \item Résumé des contributions
    \item Limites et perspectives
\end{itemize}

\end{enumerate}


\section{Développement technique}

\begin{enumerate}
    \item Introduction
          \begin{itemize}
              \item Contexte du projet
              \item Objectifs techniques
              \item Organisation du rapport
          \end{itemize}
    \item Analyse fonctionnelle
          \begin{itemize}
              \item Besoins exprimés
              \item Cahier des charges
              \item Analyse des risques
          \end{itemize}
    \item Conception technique
          \begin{itemize}
              \item Choix des technologies
              \item Schémas et modélisations
              \item Architecture du système
          \end{itemize}
    \item Réalisation
          \begin{itemize}
              \item Mise en œuvre des solutions
              \item Prototypage
              \item Développement logiciel/matériel
          \end{itemize}
    \item Tests et validation
          \begin{itemize}
              \item Plan de test
              \item Résultats et validation des performances
          \end{itemize}
    \item Conclusion et recommandations
          \begin{itemize}
              \item Évaluation des objectifs atteints
              \item Propositions d'amélioration
          \end{itemize}

\end{enumerate}