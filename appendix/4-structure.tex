\chapter{Structure}

La structure du rapport proposé dans ce modèle n'est qu'une proposition. Vous devez savoir l'adapter à votre projet et à vos besoins. Pour vous aider, voici quelques structures classiques que vous pouvez suivre pour rédiger votre rapport de bachelor.

\section{Structure Classique}

\begin{enumerate}[label=\arabic*.]
    \item Introduction
          \begin{enumerate}[label=\arabic{enumi}.\arabic*]
              \item Contexte général
              \item Problématique
              \item Objectifs de l'étude
              \item Hypothèses de travail
              \item Structure du rapport
          \end{enumerate}

    \item Revue de littérature / État de l'art
          \begin{enumerate}[label=\arabic{enumi}.\arabic*]
              \item Concepts clés
              \item Travaux antérieurs
              \item Limites des recherches existantes
          \end{enumerate}

    \item Méthodologie
          \begin{enumerate}[label=\arabic{enumi}.\arabic*]
              \item Description du cadre d'étude
              \item Matériaux et outils
              \item Méthodes expérimentales / Modélisation
              \item Procédures de collecte de données
          \end{enumerate}

    \item Résultats
          \begin{enumerate}[label=\arabic{enumi}.\arabic*]
              \item Analyse des données
              \item Présentation des résultats
              \item Validation des résultats
          \end{enumerate}

    \item Discussion
          \begin{enumerate}[label=\arabic{enumi}.\arabic*]
              \item Interprétation des résultats
              \item Comparaison avec les études précédentes
              \item Limites de l'étude
          \end{enumerate}

    \item Conclusion et perspectives
          \begin{enumerate}[label=\arabic{enumi}.\arabic*]
              \item Synthèse des résultats
              \item Recommandations
              \item Perspectives de recherche
          \end{enumerate}
\end{enumerate}

\section{Ingénierie appliquée}

En ingénierie appliquée, lors notamment du développement d'un produit, il convient généralement de suivre une structure plus technique incluant une analyse des besoins.

\begin{enumerate}[label=\arabic*.]
    \item Introduction générale
          \begin{enumerate}[label=\arabic{enumi}.\arabic*]
              \item Contexte industriel
              \item Justification du projet
              \item Objectifs spécifiques
              \item Plan du rapport
          \end{enumerate}
    \item Analyse des besoins
          \begin{enumerate}[label=\arabic{enumi}.\arabic*]
              \item Cahier des charges
              \item Spécifications techniques
              \item Analyse des contraintes
          \end{enumerate}
    \item Conception et développement
          \begin{enumerate}[label=\arabic{enumi}.\arabic*]
              \item Choix des matériaux et des technologies
              \item Design et architecture système
              \item Implémentation technique
          \end{enumerate}
    \item Tests et validation
          \begin{enumerate}[label=\arabic{enumi}.\arabic*]
              \item Méthodologie de test
              \item Résultats expérimentaux
              \item Évaluation des performances
          \end{enumerate}
    \item Discussion et recommandations
          \begin{enumerate}[label=\arabic{enumi}.\arabic*]
              \item Analyse critique des résultats
              \item Propositions d'amélioration
          \end{enumerate}
    \item Conclusion
          \begin{enumerate}[label=\arabic{enumi}.\arabic*]
              \item Résumé des contributions
              \item Limites et perspectives
          \end{enumerate}
\end{enumerate}

\newpage

\section{Développement technique}

Il peut être parfois bienvenu d'inclure une analyse fonctionnelle ainsi qu'un plan de test et de validation.

\begin{enumerate}[label=\arabic*.]
    \item Introduction
          \begin{enumerate}[label=\arabic{enumi}.\arabic*]
              \item Contexte du projet
              \item Objectifs techniques
              \item Organisation du rapport
          \end{enumerate}
    \item Analyse fonctionnelle
          \begin{enumerate}[label=\arabic{enumi}.\arabic*]
              \item Besoins exprimés
              \item Cahier des charges
              \item Analyse des risques
          \end{enumerate}
    \item Conception technique
          \begin{enumerate}[label=\arabic{enumi}.\arabic*]
              \item Choix des technologies
              \item Schémas et modélisations
              \item Architecture du système
          \end{enumerate}
    \item Réalisation
          \begin{enumerate}[label=\arabic{enumi}.\arabic*]
              \item Mise en œuvre des solutions
              \item Prototypage
              \item Développement logiciel/matériel
          \end{enumerate}
    \item Tests et validation
          \begin{enumerate}[label=\arabic{enumi}.\arabic*]
              \item Plan de test
              \item Résultats et validation des performances
          \end{enumerate}
    \item Conclusion et recommandations
          \begin{enumerate}[label=\arabic{enumi}.\arabic*]
              \item Évaluation des objectifs atteints
              \item Propositions d'amélioration
          \end{enumerate}

\end{enumerate}

\newpage

\section{Position des autres éléments}

À titre d'information voici la position habituelle des éléments suivants dans un document technique ou académique :

\begin{description}
    \item[Table des matières] Début du document, après la page de titre et les remerciements
    \item[Liste des figures] Juste après la table des matières, avant l'introduction
    \item[Liste des tables] Juste après la liste des figures
    \item[Liste des codes sources] Après la liste des tables (si applicable)
    \item[Termes et définitions] Avant l'introduction, souvent après la liste des codes
    \item[Acronymes] Avant l'introduction, souvent après les termes et définitions
    \item[Glossaire] Avant ou après les acronymes, selon la préférence
    \item[Abstract (Résumé)] Au début, après la page de titre et avant la table des matières
    \item[Corps du document] Après l'introduction, suivant l'ordre des sections principales
    \item[Bibliographie] À la fin du document, avant l'index et les annexes
    \item[Annexes] Après la bibliographie
    \item[Index] À la toute fin, après les annexes
\end{description}