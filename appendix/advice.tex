\chapter{Conseils}

\section{Adapter votre modèle}
Ce document n'est qu'un modèle ayant pour but de revoir les quelques avantages de \LaTeX~ et les fonctionnalités qui pourraient vous être utiles pour rédiger un rapport académique. N'hésitez pas à supprimer les parties inutiles et à adapter ce modèle à vos besoins.

\section{Rapport académique ou professionnel ?}

Est-ce que votre rapport est destiné à montrer à votre professeur que vous maîtrisez votre domaine et que vous êtes prêt à devenir ingénieur ou est-ce qu'il est destiné lui transmettre de nouvelles connaissances dans le domaine ?

Évidemment cet exercice académique, surtout en bachelor et en master est biaisé. Vous êtes en train de rédiger un rapport pour un professeur qui est censé vous noter et qui est payé pour vous enseigner des choses. Il sera par conséquent plus intéressant pour lui de voir que vous avez compris les concepts qu'il vous a enseigné et que vous êtes capable de les appliquer. En outre, il sera avec vous plus clément sur vos erreurs et vos approximations. En revanche dans l'industrie, ceux qui liront votre rapport n'auront pas la même sensibilité et seront plus exigeants sur la qualité de votre travail. La lecture de votre rapport sera davantage une contrainte, et le temps que vous aurez à disposition pour le rédiger sera plus court.

Il est donc important de faire la part des choses et de choisir votre style de rédaction en fonction de votre audience. N'hésitez pas à discuter de ce qu'attend votre professeur. Certains ont des attentes plus académiques d'autres plus industrielles.

\section{Quelle est la partie la plus importante ?}

Dans une entreprise, si votre travail sucite de l'intérêt dans les hautes instances et que votre rapport fini sur le bureau du directeur, il est très probable qu'il ne lise que le résumé et apprécie l'épaisseur du rapport. À l'échelon inférieur votre premier manageur lira certainement l'introduction et la conclusion. Il est donc important de soigner ces deux parties et faire en sorte qu'elles permettent de comprendre rapidement votre travail : pourquoi vous l'avez fait, comment vous l'avez fait et quels sont les résultats. Enfin, vos collègues et vos futurs collègues auront certainement de l'intérêt à lire votre travail plus en profondeur.

\section{Style de la rédaction}

Votre rapport est votre vitrine. Il sera publié sur le site de l'école, il sera diffusé sur internet et il est fort probable que l'un de vos futur employeur arrive à mettre la main dessus. Il est donc important de soigner la rédaction de votre rapport. Voici quelques conseils pour vous aider à rédiger un rapport de qualité :

\subsection{Orthographe et grammaire}

Relisez votre rapport, faites-le relire par un tiers. Les fautes d'orthographe et de grammaire sont rédhibitoires. Elles donnent une mauvaise image de vous et de votre travail. Utilisez un correcteur orthographique, mais ne vous en remettez pas entièrement à lui. Il ne corrigera pas toutes les fautes.

L'un des meilleur correcteur actuel est Druide Antidote. Il est payant mais il est très efficace. Il est disponible sur Windows, macOS et Linux et surtout il existe une extensions pour Visual Studio Code.

\subsection{Style}

Un rapport ne s'écrit généralement pas à la première personne. Préférez le style impersonnel. Ne dites pas : "J'ai réalisé ceci", mais "Ceci a été réalisé". Évitez par ailleurs les phrases trop littéraires, il est important que votre écrit soit fluide et agréable à lire mais il ne doit pas s'écarter du but de votre rapport, lequel est de comprendre votre travail le plus rapidement possible ?