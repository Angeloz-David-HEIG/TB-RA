\chapter{C'est quoi une annexe ?}

Il est important de préciser d'emblée que les annexes n'ont pas de valeur \underline{normative}, mais bien \underline{descriptive}. Elles ne doivent en aucun cas contenir des informations indispensables à la compréhension de votre travail. Leur rôle est de fournir des éléments complémentaires qui enrichissent le rapport sans en alourdir la lecture principale.

\section{Contenu des annexes}

Les annexes permettent d'approfondir certains aspects techniques ou méthodologiques de votre travail. Elles offrent au lecteur la possibilité de consulter des documents de référence ou des détails supplémentaires sans interrompre le fil principal de votre argumentation.

Voici des exemples courants de contenu pertinent pour les annexes :

\begin{itemize}
    \item Les dessins mécaniques (plans et mises en page techniques) ;
    \item Les schémas électriques détaillés ;
    \item Des photographies illustrant le projet ou les installations ;
    \item Des scripts ou extraits de code source utilisés lors de vos travaux ;
    \item Des documents techniques, tels que des fiches techniques (datasheets) ;
    \item Des démonstrations mathématiques ou des développements algébriques avancés.
\end{itemize}

\section{Rôle et importance des annexes}

Les annexes ont pour but principal de permettre à un lecteur averti de reproduire votre travail ou de l'analyser en profondeur. Elles doivent apporter une valeur ajoutée en offrant des détails qui n'ont pas leur place dans le corps principal du texte, mais qui restent essentiels pour une compréhension technique complète.

\section{Volume des annexes}

Un excès d'informations annexées peut nuire à la qualité de votre rapport. Les annexes ne doivent pas devenir un fourre-tout. Il est mal perçu d'accompagner un rapport de 50 pages avec plus de 200 pages d'annexes, car cela suggère un manque de sélection rigoureuse des informations.

Pour les ressources volumineuses, privilégiez les liens vers des ressources en ligne et assurez-vous de mentionner clairement vos sources. Si vous incluez du code source, indiquez vos dépôts (GitHub, GitLab, Bitbucket, etc.) et fournissez, si possible, le hash des commits pour permettre aux lecteurs d'accéder à la version exacte utilisée lors de la rédaction du rapport.