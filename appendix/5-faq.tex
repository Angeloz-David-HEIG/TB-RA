\chapter{FAQ}

\section{Ma compilation est trop lente}

Il est vivement recommandé d'utiliser un environnement Linux (WSL2 depuis Windows ou un Linux/Unix natif) pour profiter de la rapidité du système de fichier. Votre compilation sera beaucoup plus rapide. N'oubliez pas si vous êtes dans WSL2 de ne pas travailler depuis votre point de montage Windows (\verb!/mnt/c/Users/...!) mais depuis le système de fichier Linux (\verb!/home/user/...!).

\section{J'aimerais rajouter mon nom en haut de toutes les pages}

Il n'est généralement pas recommandé de mettre son nom sur toutes les pages d'un livre ou d'un rapport de thèse bien que de nombreux modèles le fassent et que certains enseignants le demandent. Néanmoins il existe des conventions académiques et éditoriales qui réfutent cette pratique.

\begin{itemize}
    \item Le nom de l'auteur apparaît habituellement sur la page de couverture et éventuellement dans les en-têtes des chapitres, mais pas sur chaque page.
    \item Ce serait considéré comme inutile et redondant, car le lecteur sait déjà qui a écrit le livre.
\end{itemize}

Les bonnes pratiques de mise en page recommandent de ne pas surcharger les pages de texte inutile. Les en-têtes et les pieds de page sont généralement réservés aux informations utiles pour la navigation dans le document, telles que le titre du chapitre en cours, le numéro de page, etc.

\section{Comment obtenir de l'aide sur LaTeX ?}

La meilleure ressource est \url{https://www.overleaf.com/learn}. Overleaf est un éditeur en ligne de documents LaTeX qui propose une documentation très complète et des exemples pour vous aider à démarrer. Vous pouvez également consulter le site \url{https://tex.stackexchange.com/} qui est une mine d'or pour les questions et réponses sur LaTeX.

Certains professeurs de la HEIG-VD connaissent très bien LaTeX et peuvent vous aider. N'hésitez pas à leur demander de l'aide.

\section{Que faire si j'ai une erreur de compilation ?}

Lorsque vous exécutez \verb!make! vous pouvez parfois avoir une erreur de compilation comme :

\begin{verbatim}
./appendix/examples.tex:125: LaTeX Error: Environment subfigure undefined.

See the LaTeX manual or LaTeX Companion for explanation.
Type  H <return>  for immediate help.
    ...

l.125     \begin{subfigure}
                            {0.45\textwidth}
?
\end{verbatim}

Dans ce cas, il vous faut chercher dans le fichier \verb!appendix/examples.tex! à la ligne 125 pour voir ce qui ne va pas. Le prompt \verb!?! vous permet de continuer la compilation ou de quitter. Pour continuer la compilation, tapez \verb!H <return>!. Pour quitter, tapez \verb!X <return>!.