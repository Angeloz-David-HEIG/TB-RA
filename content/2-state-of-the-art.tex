\chapter{État de l'art}

L'état de l'art constitue une revue exhaustive de la littérature existante sur le sujet de votre travail. Il s'agit d'exposer les recherches antérieures menées dans ce domaine, en présentant les solutions déjà proposées, les technologies mises en œuvre, les méthodes employées, ainsi que les résultats obtenus. Cette section a pour objectif de positionner votre étude par rapport aux connaissances actuelles et de démontrer en quoi votre démarche apporte une contribution originale. Elle permet ainsi de mettre en lumière les limites des travaux existants et de justifier la pertinence de votre recherche.

\section{Travaux existants}

Dans cette section, vous pouvez exposer de manière structurée les travaux antérieurs relatifs à votre sujet. Il peut être pertinent d'organiser ces contributions selon différents critères : par thématique, par approche méthodologique, par auteur ou encore par chronologie.

Il est impératif de citer rigoureusement vos sources et de vous conformer aux normes de citation en vigueur. Chaque référence doit figurer dans la bibliographie afin de garantir la crédibilité de votre propos et de respecter les exigences éthiques liées à la recherche scientifique.

\section{Technologies et méthodes existantes}

Cette section a pour vocation de présenter les technologies et les méthodes déjà développées pour traiter la problématique que vous abordez. Il est important d'examiner et d'analyser de manière critique ces approches : quelles sont leurs forces ? Leurs limites ? Quels résultats ont-elles permis d'obtenir ?

Il ne s'agit pas de rédiger un cours magistral, mais plutôt de proposer une synthèse pertinente et ciblée des connaissances existantes, en rapport direct avec votre sujet. Il peut être particulièrement utile d'inclure des éléments visuels tels que des tableaux comparatifs, des graphiques ou des schémas, afin de faciliter la compréhension des comparaisons et des analyses.

Évitez de vous contenter d'importer des illustrations trop complexes ou inadaptées, trouvées sur Internet. N'hésitez pas à créer vos propres figures pour qu'elles servent précisément votre propos et s'intègrent harmonieusement à votre raisonnement.

Rappelez-vous que cette section doit progressivement amener le lecteur à comprendre les enjeux de votre travail. Vous devez poser, avec méthode, les bases de votre réflexion et conduire le lecteur à saisir les éléments essentiels qui éclaireront, tout au long de votre étude, la pertinence et la singularité de votre démarche.