\chapter{État de l'art}

% \index{état de l'art}
% \index{méthode}
% L'état de l'art constitue une revue exhaustive de la littérature existante sur le sujet de votre travail. Il s'agit d'exposer les recherches antérieures menées dans ce domaine, en présentant les solutions déjà proposées, les technologies mises en œuvre, les méthodes employées, ainsi que les résultats obtenus. Cette section a pour objectif de positionner votre étude par rapport aux connaissances actuelles et de démontrer en quoi votre démarche apporte une contribution originale. Elle permet ainsi de mettre en lumière les limites des travaux existants et de justifier la pertinence de votre recherche.

\section{Travaux existants}

Le seul travail existant est celui de Mme. Di Luca cité en \ref{Context}, qui est le travail de bachelor cité dans  Au moment de l'écriture du rapport
intermédiaire, rien n'a été emprunté à ce travail. \textit{Les éléments de gestion d'énigme pourront être intégrés ou modifiés pour accélérer la partie gestion
    d'énigmes du présent travail, dans ce cas le texte présent sera modifié pour refléter ce cas dans le rapport final.}
% Dans cette section, vous pouvez exposer de manière structurée les travaux antérieurs relatifs à votre sujet. Il peut être pertinent d'organiser ces contributions selon différents critères : par thématique, par approche méthodologique, par auteur ou encore par chronologie.

% Il est impératif de citer rigoureusement vos sources et de vous conformer aux normes de citation en vigueur. Chaque référence doit figurer dans la bibliographie afin de garantir la crédibilité de votre propos et de respecter les exigences éthiques liées à la recherche scientifique.

\section{Technologies et méthodes existantes}
\subsection{Unity}
"Unity est un moteur de jeu multiplateforme (smartphone, ordinateur, consoles de jeux vidéo et Web) développé par Unity Technologies.
Il a la particularité de proposer une licence gratuite dite « Personal » avec quelques limitations de technologie avancée au niveau de l'éditeur,
mais sans limitation au niveau du moteur.

Il est l'un des deux moteurs les plus répandus dans l'industrie du jeu vidéo,
aussi bien pour les grands studios que pour les indépendants. Par rapport à son concurrent principal,
Unreal Engine, Unity est considéré comme ayant une interface utilisateur plus facile d'accès et un service plus approprié pour les créations indépendantes.
"\cite{UnityMoteurJeu2025}

Unity permet le développement d'application pour Android et iOS en réalité augmentée grâce à ARCore (Google, Android), ARKit (Apple, iOS), ce dernier ne sera pas mentionné car le
développement de l'application sur iOS a été mis de côté pour des raisons de difficulté (Apple impose l'utilisation d'un Mac et de XCode qui rendent le déploiement initial sur iOS plus
compliqué que celui sur Android). \textit{Un des objectifs facultatifs et d'avoir l'application disponible sur iOS. Le rapport final pourrait avoir atteint cet objectif.}

\subsection{ARCore}
ARCore est le \acrshort{sdk} développé par Google. Il permet de concevoir des applications en réalité augmentée. C'est le module principal qui est utilisé dans Unity pour produire l'application du travail.
A cela vient s'ajouter ARCore Extensions qui permet d'utiliser Google Geospatial ainsi que Google Geospatial Creator.
\subsection{Google Geospatial (Creator)}
Google Geospatial est le noyau pour les applications basées sur une localisation de Google, Geospatial Creator est un outil dans Unity qui permet de placer des objets virtuels par rapport à
des coordonnées du monde réel (position GPS, altitude). \acrshort{ggc} requiert également l'utilisation de Cesium pour bien fonctionner


\subsection{Cesium}
Cesium est la plateforme qui va permettre de \textit{\acrshort{rdr}} les textures des bâtiments, objets, etc. de la réalité dans le moteur Unity.
Dans les menus d'Unity, c'est plus spécifiquement Cesium Ion qui est utilisé.
% \index{technologie}
% Cette section a pour vocation de présenter les technologies et les méthodes déjà développées pour traiter la problématique que vous abordez. Il est important d'examiner et d'analyser de manière critique ces approches : quelles sont leurs forces ? Leurs limites ? Quels résultats ont-elles permis d'obtenir ?

% Il ne s'agit pas de rédiger un cours magistral, mais plutôt de proposer une synthèse pertinente et ciblée des connaissances existantes, en rapport direct avec votre sujet. Il peut être particulièrement utile d'inclure des éléments visuels tels que des tableaux comparatifs, des graphiques ou des schémas, afin de faciliter la compréhension des comparaisons et des analyses.

% Évitez de vous contenter d'importer des illustrations trop complexes ou inadaptées, trouvées sur Internet. N'hésitez pas à créer vos propres figures pour qu'elles servent précisément votre propos et s'intègrent harmonieusement à votre raisonnement.

% Rappelez-vous que cette section doit progressivement amener le lecteur à comprendre les enjeux de votre travail. Vous devez poser, avec méthode, les bases de votre réflexion et conduire le lecteur à saisir les éléments essentiels qui éclaireront, tout au long de votre étude, la pertinence et la singularité de votre démarche.