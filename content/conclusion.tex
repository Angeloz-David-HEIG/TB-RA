\chapter{Conclusion}

On dénote deux types de conclusion : la conclusion technique et la conclusion personnelle. La conclusion est la deuxième partie qui sera lue par un manager de société après le résumé. Elle doit être concise et claire. Elle doit résumer les résultats obtenus et dresser une conclusion objective du projet. La conclusion technique doit être basée sur les résultats obtenus et doit être en lien avec les objectifs fixés. La conclusion personnelle est une réflexion personnelle sur le projet. Elle peut contenir des éléments qui n'ont pas été abordés dans le rapport, des éléments qui ont été appris durant le projet, des éléments qui ont été appréciés ou non, etc.

Bien entendu vous pouvez également donner un sens plus intime en parlant de votre ressenti, de ce que vous avez appris, de ce que vous avez aimé ou non, etc. C'est une manière de donner une touche personnelle à votre travail.

Il peut être coutume de signer la conclusion encore que votre rapport comporte déjà votre signature numérique. Cela reste un choix personnel.

\vfil
\hspace{8cm}\makeatletter\@author\makeatother\par
\hspace{8cm}\begin{minipage}{5cm}
    % Place pour signature numérique
    \printsignature
\end{minipage}