\chapter{Implémentation}
\section{Formation sur Unity}
Afin de se familiariser avec Unity, car il n'en avait aucune connaissance avant le début de ce projet, l'étudiant a suivi deux tutoriels disponibles
sur le site du logiciel\cite{UnityEssentialsPathway}\Cite{MobileARDevelopment}. Le premier est \textit{Unity Essentials}, il présente :

\begin{itemize}[label={\textbullet}]
    \item L'installation d'Unity
    \item La création de projet avec ou sans template.
    \item Une explication de l'interface de l'éditeur d'Unity ainsi que ses commandes principales.
    \item Une introduction aux environnement 2D et 3D.
    \item Une introduction à l'ajout d'audio.
    \item Une introduction à la programmation (scripts).
    \item Le déploiement sur WebGL d'une application.
\end{itemize}


Le deuxième tutoriel est spécifique à la réalité augmentée, \textit{Mobile AR Development} et explique comment :

\begin{itemize}[label={\textbullet}]
    \item Créer un filtre pour visage (comme Snapchat par exemple).
    \item Détecteur un repère (QR Code, autre) puis faire apparaître un objet dessus.
    \item Détecter un plan en RA, puis faire apparaître un objet en cliquant sur l'écran du téléphone.
    \item Déployer une application sur smartphone.
\end{itemize}

\section{Implémentation de l'application}
\subsection{Déploiement de Geospatial Creator dans Unity}

\subsection{Test de l'application d'exemple de Google}

\subsection{Ajout d'un objet virtuel proche de l'étudiant}

\subsection{Interaction avec l'objet}

\subsection{Choix du parcours}
Pour avoir un parcours qui se place dans du concret, les transports lausannois
ont été contactés dans le but de leur proposer une collaboration, l'étudiant ayant pensé à
utiliser un parcours, qui fait se déplacer l'utilisateur dans toute la ville de Lausanne,
créé en 2021 pour les 125 ans des tl.\cite{hatetVeriteLionsLausannois2021} Malheureusement, aucune réponse n'a été reçue de leur
part pour l'instant. Un parcours alternatif est créé par l'étudiant pour palier l'absence de
cette collaboration
% Ce chapitre est consacré à la description détaillée de l'implémentation de votre projet. Il a pour objectif d'expliquer, de manière claire et méthodique, comment vous avez concrètement mis en œuvre votre travail. Vous y présenterez, étape par étape, le processus qui a conduit à la réalisation de votre projet, en justifiant les choix techniques et méthodologiques effectués.

% \index{expérience}
% Vous devez décrire le déroulement de vos expériences, en précisant le cadre dans lequel elles ont été menées ainsi que les outils et technologies utilisés. Ce chapitre doit permettre au lecteur de comprendre non seulement ce que vous avez fait, mais aussi pourquoi vous avez opté pour cette approche.

% Exposez de manière structurée la conception globale de votre projet en illustrant vos explications par des schémas, des diagrammes ou des modèles visuels -- tels que des diagrammes de classes, de flux ou des organigrammes -- afin de clarifier votre démarche.

% Que ce soit pour une architecture logicielle, une conception électronique ou une analyse de données, chaque aspect de votre travail doit être présenté de façon logique et détaillée. Il est essentiel de justifier vos choix en soulignant leur pertinence au regard des objectifs fixés et de la problématique posée.

% Vous êtes libre de structurer ce chapitre en plusieurs sections distinctes, en fonction de la nature de votre projet. L'objectif est d'offrir au lecteur une vision claire de la manière dont vous avez développé votre travail, en mettant en lumière non seulement les actions entreprises, mais également les décisions qui les ont motivées. Votre implémentation doit refléter une réflexion méthodique et rigoureuse, guidée par la problématique et les objectifs initiaux.
