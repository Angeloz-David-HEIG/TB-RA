\chapter{État de l'art}

L'état de l'art présente une revue de la littérature sur le sujet de votre travail. Il s'agit de présenter les travaux antérieurs qui ont été réalisés sur le sujet, les solutions existantes, les technologies utilisées, les méthodes employées, les résultats obtenus, etc. L'état de l'art permet de situer votre travail par rapport à ce qui a déjà été fait et de justifier l'originalité de votre approche.

\section{Travaux existants}

Dans cette section, vous pourriez y présenter les travaux existants qui ont été réalisés sur le sujet de votre travail. Vous pourriez les classer par thème, par auteur, par date, etc. N'oubliez pas que vous devez citer les sources de manière précise et les référencer dans la bibliographie.

\section{Technologies et méthodes existantes}

Dans cette section, vous pourriez y présenter les technologies et les méthodes connues qui ont été utilisées pour résoudre le problème que vous vous apprêtez à résoudre. Vous pourriez les comparer, les évaluer, les critiquer, etc. N'oubliez pas que vous devez citer les sources de manière précise et les référencer dans la bibliographie.

L'objectif n'est pas de faire un cours magistral sur le sujet, mais de donner un aperçu des connaissances existantes sur le sujet qui servent le propos que vous vous apprêtez à développer. Il peut être utile de présenter des tableaux comparatifs, des graphiques, des schémas, etc.

Évitez d'aller à la pioche de figures trop détaillées sur internet. N'hésitez pas à les dessiner vous-même pour les adapter à votre propos. N'oubliez pas que vous racontez une histoire, vous posez peu à peu les bases de votre raisonnement et vous amenez le lecteur saisir les éléments clés qui lui permettront de comprendre au fil de la lecture votre travail.