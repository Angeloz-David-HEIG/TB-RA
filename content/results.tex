\chapter{Analyse des résultats}

Vous avez réalisé votre projet, vous avez mené vos expériences, vous avez collecté vos données, vous avez analysé vos résultats. Il est temps de les présenter et de les interpréter. Cette section est cruciale pour la validation de votre travail. Vous devez montrer que vous avez atteint vos objectifs, que vous avez répondu à vos questions de recherche, que vous avez résolu le problème que vous vous étiez fixé.

Prenez garde à ne pas surcharger votre rapport de données brutes qui ne servent pas l'objectif final de votre projet. Avec tout le temps que vous avez passé sur votre projet, vous avez probablement collecté une quantité impressionnante de données, mais la plupart n'ont d'importance que parce que vous les avez collectées. Vous devez sélectionner les données qui sont pertinentes.

\section{Présentation des résultats}

Généralement on commence par présenter les résultats de manière synthétique. Vous pouvez utiliser des tableaux, des graphiques, des schémas, des figures, des photos, des extraits de code, etc., pour illustrer vos résultats. Vous devez les commenter, les expliquer, les interpréter.

Pensez à aux unités, aux biais, aux incertitudes.

\section{Discussion des résultats}

Dans un second temps, vous pouvez discuter vos résultats. Vous pouvez les comparer à ceux de la littérature, les évaluer, les critiquer, les justifier. Vous pouvez expliquer pourquoi vous avez obtenu ces résultats, pourquoi ils sont pertinents.
