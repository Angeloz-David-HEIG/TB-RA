\begin{abstract}

    % Francais
    Le résumé du projet doit être rédigé en premier lieu dans la langue principale du document. Il s'agit d'un élément essentiel, visant à fournir une synthèse claire, concise et informative du contenu du travail. Sa longueur idéale se situe entre 150 et 300 mots. Il se présente généralement sous la forme d'un seul paragraphe, sans mise en forme particulière ; pas de listes, de citations ou de références bibliographiques. Le résumé doit être rédigé de manière objective et factuelle, en employant soit le présent -- pour énoncer des faits généraux et des résultats établis -- soit le passé : pour décrire les démarches entreprises et les résultats obtenus. L'usage du futur est à proscrire, car le projet est considéré comme terminé et aucun résultat à venir ne doit être évoqué. Son objectif principal est de résumer efficacement le travail effectué, en mettant en avant le contexte et les objectifs du projet ; la méthodologie ou les approches utilisées ; les résultats principaux obtenus ; ainsi que les conclusions ou recommandations majeures. Ce résumé, appelé \emph{abstract} en anglais, doit rester neutre et sans opinion personnelle. Il doit permettre au lecteur, même sans lire l'intégralité du document, de saisir rapidement la portée, les enjeux et les apports du travail réalisé.

    %% L'asterisme est un signe typographique en forme d'étoile, utilisé pour marquer une pause dans un texte ou pour séparer des paragraphes. Il est souvent utilisé pour indiquer un changement de scène dans un récit. Bien qu'il se fasse rare dans la typographie moderne, c'est un symbole de choix pour séparer les différentes langues du résumé de thèse.
    \asterism

    % English
    The project abstract must be written first in the document's main language. It is an essential element, intended to provide a clear, concise, and informative summary of the work's content. Its ideal length is between 150 and 300 words. It is generally presented as a single paragraph, without any special formatting; no lists, citations, or bibliographic references. The abstract should be written in an objective and factual manner, using either the present tense -- to state general facts and established results -- or the past tense: to describe the steps taken and the results obtained. The use of the future tense is prohibited, as the project is considered complete and no forthcoming results should be mentioned. Its main objective is to effectively summarize the work carried out, highlighting the context and objectives of the project; the methodology or approaches used; the main results obtained; and the key conclusions or recommendations. This abstract must remain neutral and free of personal opinion. It should allow the reader, even without reading the entire document, to quickly grasp the scope, challenges, and contributions of the work presented.

\end{abstract}