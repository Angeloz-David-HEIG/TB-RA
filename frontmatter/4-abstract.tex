\begin{abstract}

    % Français
    Un travail de bachelor réalisé en 2024 avait pour objectif d'inciter les jeunes à se rendre plus souvent dans les musées en leur offrant une expérience sur smartphone ou tablette. Le travail a étudié la faisabilité d'une application permettant de localiser un joueur à l'intérieur d'un musée, de l'orienter et de lui proposer des énigmes. Le travail actuel consiste à étendre cette expérience en extérieur et d'utiliser les capteurs intégrés à un smartphone
    % Le résumé du projet doit être rédigé en premier lieu dans la langue principale du document. Il s'agit d'un élément essentiel, visant à fournir une synthèse claire, concise et informative du contenu du travail. Sa longueur idéale se situe entre 150 et 300 mots. Il se présente généralement sous la forme d'un seul paragraphe, sans mise en forme particulière ; pas de listes, de citations ou de références bibliographiques. Le résumé doit être rédigé de manière objective et factuelle, en employant soit le présent -- pour énoncer des faits généraux et des résultats établis -- soit le passé : pour décrire les démarches entreprises et les résultats obtenus. L'usage du futur est à proscrire, car le projet est considéré comme terminé et aucun résultat à venir ne doit être évoqué. Son objectif principal est de résumer efficacement le travail effectué, en mettant en avant le contexte et les objectifs du projet ; la méthodologie ou les approches utilisées ; les résultats principaux obtenus ; ainsi que les conclusions ou recommandations majeures. Ce résumé, appelé \emph{abstract} en anglais, doit rester neutre et sans opinion personnelle. Il doit permettre au lecteur, même sans lire l'intégralité du document, de saisir rapidement la portée, les enjeux et les apports du travail réalisé.
    %% l’astérisme est un signe typographique en forme d'étoile, utilisé pour marquer une pause dans un texte ou pour séparer des paragraphes. Il est souvent utilisé pour indiquer un changement de scène dans un récit. Bien qu'il se fasse rare dans la typographie moderne, c'est un symbole de choix pour séparer les différentes langues du résumé de thèse.


\end{abstract}