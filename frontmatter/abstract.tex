%% Le résumé du projet doit:
%% 1. Être rédigé dans la langue du rapport
%% 2. Optionnellement offrir une traduction anglaise
%% 3. Être court, concis et informatif (entre 150-300 mots)
%% 4. Être rédigé en un seul paragraphe, sans mise en forme (gras ou italique)
%% 5. Être au passé (le projet étant terminé, on résume ce qui a été fait et les résultats obtenus)

% Francais
Le résumé du projet doit être rédigé en premier dans la langue principale du document. Il doit être court -- 150 à 300 mots -- concis et informatif. Il se compose généralement d'un seul paragraphe, sans mise en forme particulière. Il est de coutume de le rédiger soit au présent soit au passé, et le futur est proscrit, puisque le projet est terminé, il n'y a pas de résultats à venir. Le résumé appelé \emph{abstract} en anglais résume donc le travail effectué, les résultats obtenus et les conclusions tirées sans opinion personnelle.

%% L'asterisme est un signe typographique en forme d'étoile, utilisé pour marquer une pause dans un texte ou pour séparer des paragraphes. Il est souvent utilisé pour indiquer un changement de scène dans un récit. Bien qu'il se fasse rare dans la typographie moderne, c'est un symbole de choix pour séparer les différentes langues du résumé de thèse.
\asterism

% English
The project summary must be written first in the document's primary language. It should be short -- 150 to 300 words -- concise, and informative. It generally consists of a single paragraph, without any special formatting. It is customary to write it either in the present or past tense; the future tense is not allowed, as the project is completed and there are no forthcoming results. The summary, known as the abstract in English, thus outlines the work carried out, the results obtained, and the conclusions drawn, without any personal opinion.
